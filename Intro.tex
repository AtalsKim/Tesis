\documentclass[main]{subfiles}
\begin{document}

\section*{Introducción}
\addcontentsline{toc}{chapter}{Introducción}
Aproximadamente más de un millón de personas dependen diariamente del Metro de Caracas para transportarse dentro de la zona metropolitana. Dado que la continuidad de la prestación de servicio por parte de esta compañía de transporte es vital para la ciudad, cualquier irregularidad en el contacto mecánico rueda-riel debe ser detectada a tiempo, por lo que la evaluación de la calidad de este contacto es significativamente importante para la empresa. 
 
En este trabajo se presentan los resultados de la investigación que se llevó a cabo para  desarrollar un instrumento que permita evaluar ese contacto mecánico entre la rueda y el riel en el servicio de trenes comerciales de la línea 1 del Metro de Caracas C.A.
 
En el primer capítulo se expone el contexto del problema del comportamiento contacto mecánico rueda-riel y la hipotesis de elaboración de un instrumento de evaluación de ese contacto rueda-riel. De igual manera se identifican los objetivos generales y específicos de la investigación y la justificación que respalda la necesidad de este trabajo.
 
El segundo capítulo ofrece el marco referencial que enmarca esta investigación. Expone el estado de los estudios realizados en lo que se refiere al tema del contacto rueda-riel y la base teórica en la que se fundamenta el contacto mecánico de Hertz y los modelos de contacto tangencial de Kalker, Shen et al y Polanch. Así mismo se identifican los fundamentos teóricos de los equipos e instrumentos utilizados en este trabajo.

En lo referente al marco metodológico, tema tratado en el tercer capítulo, se describe el tipo de estudio realizado y el diseño que se adoptó para dar respuestas a los interrogantes de la investigación sobre el contacto rueda-riel. Se explica brevemente la instrumentación utilizada para el estudio del contacto mecánico, detallando sus componentes e identificando el modo de trabajo con el que se trabajó.  Además, se describen las técnicas aplicadas en las distintas etapas de la investigación y el procedimiento para el análisis de la data recolectada bajo los modelos de contacto tangencial expuestos por Kalker, Shen et al y Polanch.
 
En el cuarto y quinto capítulo, se exponen los datos obtenidos en pruebas realizadas en vía férrea y su análisis respectivo identificando las variables que se relacionan o influyen directamente sobre el fenómeno para compararlos con los modelos de contacto rueda-riel mencionados en el párrafo anterior. Finalmente se exponen las conclusiones que resumen los resultados obtenidos en las pruebas realizadas luego de su análisis para darle respuesta a los  objetivos que enmarcan esta investigación.
\end{document}
